\documentclass{article}
\title{Livestock Breeding and Genomics I$+$II\\Syllabus}
\author{Birgit Gredler and Peter von Rohr}

\begin{document}
\maketitle

\section*{General Information}
\begin{tabular}{p{2.5cm}p{8cm}}
Lecturers & B. Gredler (BG), P. von Rohr (PvR)\\
Date      & Fr	09-12              \\
Location  & LFW C 11               \\
\end{tabular}

\section*{Dates}
\begin{tabular}{p{2.5cm}p{8cm}}
Part  &  Dates \\
I     & 18.09., 25.09., 02.10., 16.10., 23.10., 30.10., 06.11., 13.11.\\
II    & 20.11., 27.11., 04.12., 11.12., 18.12.\\
\end{tabular}

\vspace{2ex}
\noindent No lecture on 09.10.

\vspace{2ex}
\noindent Exam: 18.12.

\section*{Topics}

\begin{tabular}{|p{1cm}|p{7cm}|p{1.5cm}|p{1.5cm}|}
\hline
Part & Topic & Date & Lecturer \\
\hline
I    & Selection index (various sources of information, one trait, multiple traits)
     & 18.09., 25.09.
     & PvR\\
\hline
I    & Relationship matrix and its inverse
     & 02.10.
     & PvR\\
\hline
I    & Correction of fixed effects
     & 16.10., 23.10.
     & PvR\\
\hline
I    & Introduction to methods for the estimation of variance components
     & 30.10., 06.11.
     & PvR\\
\hline
I    & BLUP: one trait, repeated observations, multiple traits, economic indices
     & 06.11., 13.11.
     & BG\\
\hline\hline
II   & Linkage disequilibrium
     & 20.11.
     & BG\\
\hline
II   & Genomic selection and estimation of breeding values
     & 27.11., 04.12.
     & BG\\
\hline
II   & Genomewide association studies
     & 11.12.
     & BG\\
\hline

\end{tabular}

\clearpage
\pagebreak

\section*{Weekly Schedule}

\begin{tabular}{|p{1cm}|p{1.5cm}|p{7cm}|p{1.5cm}|}
\hline
Week & Date   & Topic & Lecturer \\
\hline
$1$  & 18.09  & Introduction to the course, Selection index & PvR \\
\hline
$2$  & 25.09  & Selection index multiple traits             & PvR     \\
\hline
$3$  & 02.10  & Relationship matrix and its inverse         & PvR     \\
\hline
$4$  & 09.10  & No lecture & \\
\hline
$5$  & 16.10  & Correction of fixed effects                 & PvR     \\
\hline
$6$  & 23.10  & Anova                                       & PvR     \\
\hline
$7$  & 30.10  & Variance components estimation              & PvR     \\
\hline
$8$  & 06.11  & Variance compentents part II                & PvR     \\
     &        & BLUP one trait                              & BG      \\
\hline
$9$  & 13.11  & BLUP multiple traits, economic indices      & BG      \\
\hline
\hline
$10$ & 20.11  & Linkage disequilibrium                      & BG      \\
\hline
$11$ & 27.11  & Genomic selection                           & BG      \\
\hline
$12$ & 04.12  & Genomic selection                           & BG      \\
\hline
$13$ & 11.12  & Genomewide association studies              & BG      \\
\hline
$14$ & 18.12  & Exam                                        & BG, PvR \\
\hline
\end{tabular}

\section*{Website}
The course website is available at: \\
\verb+http://charlotte-ngs.github.io/LivestockBreedingAndGenomics/+. Slides and exercises will be made available to the students on that site.

\section*{Goals}
\subsection*{Part I}
The students are able to estimate breeding values for the most common population structures using the selection index. They are able to set up design matrices, the relationship matrix and its inverse as well as the Mixed Model equations to estimate BLUP breeding values for smaller examples.

\subsection*{Part II}
The students are able to interpret and apply linkage disequilibrium. They are able to discuss the principles of genomic selection, genome wide association studies and breeding value estimation.

\section*{Literature}
A basic introduction on animal breeding at the bachelors level is presented by Willam and Simianer \cite{Willam2011}. Burdon \cite{Bourdon1999} provides further understanding of animal breeding.

A standard introduction to quantiative Genetics is given in \cite{Falconer1996}. G\"otz, Sch\"uler and Swalve \cite{GSS2002} translated that introduction into German.

Lynch and Walsh \cite{Lynch1998} explain the basics of quantitative genetics in great detail in the first volume of their two volume series on genetics and evolution.


Mrode and Thompson \cite{Mrode2005} put linear models and their application in animal breeding at the center of their book.

\bibliographystyle{plain}
\bibliography{LBA.bib}
\end{document}
